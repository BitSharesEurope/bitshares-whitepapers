There is always concern of price manipulation. Someone with a large amount of
money on both sides of a trade can use their funds to manipulate the markets
and thus the price feed. If the amount of money they lose manipulating the
markets is less than the amount of money they can gain by manipulating the
price feed, then it will be profitable to manipulate the market at the expense
of either the BitUSD longs or the shorts. A low-collateralized short that sees
a large force-settlement order requested can attempt to manipulate the markets
and thus the feed against the BitUSD holder.

The risk of price manipulation is priced into the premium on BitUSD charged by
the shorts, and thus should already be priced into the market. If price
manipulation became a serious problem that caused very high premiums, then it
could be addressed by the price feed producers, who can adopt a moving average
over wider time windows to increase the difficulty of short-term manipulation.
A variety of algorithms could be used to estimate a \emph{fair price} that
keeps BitUSD valued at least \$1.00.

In practice, a feed producer can observe the BitUSD-to-USD market as an
indicator on which way to adjust the feed. Generally speaking, the strategy
that the feed producers adopt for controlling the feed should be public
knowledge, because the shorts will ultimately rely on it. For the feed
producers to change strategies in unpredictable ways could cause losses to both
longs and shorts. Fortunately, it is assumed that shareholders primarily
approve complying and quickly fire misbehaving feed providers.
