The first proposal of the BitAsset system has evolved over 9 months since it
first launched as we learned how market participants reacted to various rules.
Liquidity is critical to confidence in the value of the token. A system with
unbalanced rules will tend to bias the price in one direction or the other.

Early on, BitUSD was driven down to \$0.85 as demand for shorting outstripped
demand for BitUSD and shorts were not forced to cover. Then, after implementing
a \emph{30 day forced covering rules}, the price stabilized around \$0.98 to
\$1.00. Later, as the cryptocurrency bear market progressed, we had BitUSD
trading at \$1.05 or more because everyone is scared to use leverage and those
that have open positions looked to cover their position while those who held
BitUSD were not looking to sell. Over the course of these past 9 months, we
have seen 3 different markets and had an opportunity to better understand the
behavior of market participants.

In order for BitUSD to be accepted as being equal to \$1.00 for the purposes of
setting prices and online shopping, it only needs to maintain a \emph{floor} of
\$1.00. If it can maintain a floor of \$1.00, then merchants can accept it,
know their margins are safe, and that they are \emph{not exposed to currency
risk}. In order to enable a guaranteed floor, all BitUSD can be \emph{force
liquidated} at a trustworthy price feed\footnote{Price feeds are published by
\emph{delegates} that have shareholder approval. See~\cref{sec:feeds}.}. If
this rule is present, then those who create the BitUSD must sell it at a price
that properly accounts for this risk of so called \emph{forced settlement}.
This means that at almost all times new BitUSD will only enter circulation when
there is a buyer willing to pay a premium for a guaranteed floor.

As we will see, since USD holders can initiate settlement, there is no need for
artificial forced covering every 30 days. This relieves shorts of risk, helps
increase short demand, and keeps the price of BitUSD near the floor.
