BitShares makes use of read-able account names that have to be registered
together with a public key in the blockchain prior to its usage. On
registration, a new account is associated with an individual and unique
identifier that will be used internally to identify an account. In this case,
the blockchain acts as a name--to--public-key resolver similar to the
traditional domain name service (DNS). These named accounts enable users to
easily remember and communicate their account information.

%% transferable
In BitShares, an account name can be transfered by updating the public key that
is associated with it.

%% recurring
Furthermore, BitShares is the first smart contract platform with built-in
support for \emph{recurring payments} and subscription payments. This feature
allows users to authorize third parties to make withdrawals from their accounts
within certain limits. This is a convenient way to \emph{set it and forget it}
for monthly bills and subscriptions.

%% dynamic
BitShares also designs permissions around people, rather than around
cryptography, making it easy to use. Every account can be controlled by any
weighted combination of other accounts and private keys. This creates a
hierarchical structure that reflects how permissions are organized in real
life, and makes multi-user control over funds easier than ever. Multi-user
control is the single biggest contributor to security, and, when used properly,
it can virtually eliminate the risk of theft due to hacking. Hence, BitShares
does technically not have multi-\emph{signature} accounts, but has
multi-\emph{account} permissions.

\medskip

In the following we introduce the above mentioned features that come with named
account in the BitShares network in more detail.
