In BitShares, each account is separated into
\begin{itemize}
 \item \emph{Active Permission} which has control over its funds and
 \item \emph{Owner Permission} which controls the account itself.
\end{itemize}

Furthermore, BitShares uses \emph{authorities} consisting of one or many
entities that authorize an action, such as transfers, trades or account
modifications. An authority consists of one or several pairs of an account name
with a \emph{weight}. In order to obtain a valid transaction, the sum of the
weights from signing the parties has to exceed the threshold as defined in the
permissions.

Let's discuss some examples to shed some light on the used terminology and the
use-cases. We assume that a new account is created with it's active permissions
set as described below. Note that the same scheme also works for the owner
permissions!

A flat multi-signature scheme is composed of $M$ entities of which $N$ entities
must sign in order for the transaction to be valid. Now, in BitShares, we have
\emph{weights} and a \emph{threshold} instead of $M$ and $N$. Still we can
achieve the very same thing with even more flexibility as we will see now.

Let's assume, Alice, Bob, Charlie and Dennis have common funds. We want to be
able to construct a valid transaction if only two of those agree. Hence a
2-of-4 ($N$-of-$M$) scheme can look as follows:

\begin{center}
 \begin{tabular}{l|c}
  \hline
  Account             & Weight\\\hline\hline
  Alice               & 33\%\\
  Bob                 & 33\%\\
  Charlie             & 33\%\\
  Dennis              & 33\%\\\hline\hline
  \textbf{Threshold:} & 51\%\\
  \hline
 \end{tabular}
\end{center}

All four participants have a weight of 33\% but the threshold is set to 51\%.
Hence only two out of the four need to agree to validate the transaction.
Alternatively, to construct a 3-of-4 scheme, we can either decrease the weights
to 17 or increase the threshold to 99\%.

With the threshold and weights, we now have more flexibility over our funds, or
more precisely, we have more \emph{control}. For instance, we can have separate
weights for different people. Let's assume Alice wants to secure here funds
against theft by a multi-signature scheme but she does not want to hand over too
much control to her friends. Hence, we create an authority similar to:

\begin{center}
 \begin{tabular}{l|c}
  \hline
  Account             & Weight\\\hline\hline
  Alice               & 49\%\\
  Bob                 & 25\%\\
  Charlie             & 25\%\\
  Dennis              & 10\%\\\hline\hline
  \textbf{Threshold}: & 51\%\\
  \hline
 \end{tabular}
\end{center}

Now the funds can either be accessed by Alice and a single friend or by all
three friends together.

Let's take a look at a simple multi-hierarchical corporate account setup.  We
are looking at a company that has a Chief of Financial Officer (CFO) and a some
departments working for him, such as the Treasurer, Controller, Tax Manager,
Accounting, etc. The company also has a CEO that wants to have spending
privileges. Hence we construct an authority for the funds according to:

\begin{center}
 \begin{tabular}{l|c}
  \hline
  Account               & Weight\\\hline
  CEO.COMPANY           & 51\%\\
  CFO.COMPANY           & 51\%\\\hline
  \textbf{Threshold}:   & 51\%\\
  \hline
 \end{tabular}
\end{center}

whereas CEO.COMPANY and CFO.COMPANY have their own authorities. For instance,
the CFO.COMPANY account could look like:

\begin{center}
 \begin{tabular}{l|c}
  \hline
   CFO.COMPANY             & Weight\\\hline
   Chief.COMPANY           & 51\%  \\
   Treasurer.COMPANY       & 33\%  \\
   Controller.COMPANY      & 33\%  \\
   Tax Manager.COMPANY     & 10\%  \\
   Accounting.COMPANY      & 10\%  \\\hline
   \textbf{Threshold}:     & 51\%  \\
  \hline
 \end{tabular}
\end{center}

This scheme allows:

\begin{itemize}
 \item the CEO to spend funds
 \item the Chief of Finance Officer to spend funds
 \item Treasurer together with Controller to spend funds
 \item Controller or Treasurer together with wither the Tax Manager or Accounting to
       spend funds.
\end{itemize}

Hence, a try of arbitrary depth can be spanned in order to construct a flexible
authority to reflect mostly any business use-case.
