\label{sec:mpa}
%%%%%%%%%%%%%%%%
% Smartcoins
%%%%%%%%%%%%%%%%
In today's world, crypto-currencies are unique because they are the only type
of digital currency that does not represent a corresponding counterparty
liability. Instead, they are \emph{fungible} \emph{decentralized} tokens, whose
value is derived from the amount of practical utility (or potential future
utility) perceived by the network of users that support and trade in them.

Not surprisingly, most cryptocurrencies suffer from high levels of price
volatility due to many complex factors, such as constantly shifting
public perception and highly speculative and unregulated markets.
Although professional traders tend to appreciate this volatility, so far
it has hindered the widespread adoption of cryptocurrency as a
\emph{practical payment solution}. A cryptocurrency that has the
properties and advantages of Bitcoin, but is also capable of maintaining
price parity with a globally adopted currency (e.g. U.S. dollar), would
have incredible high utility for convenient and fast e-commerce.

The BitShares Blockchain tries to solving this problem by introducing
smartcoins - a framework for collateralized counterpartyrisk-free loans
that enable token that highly correlate with the valuation of an
underlying asset, such as U.S. Dollar.

% Terminology
In this paper we refer to \emph{smartcoins} as a means for customization of
tokens within a \emph{framework} of smart contracts and parameters as offered
by the BitShares Blockchain. In literature, smartcoins are also often referred
to as \emph{bitassets} or \emph{market pegged asset}.

As such, tokens like \emph{bitUSD}, \emph{bitCNY} and others are merely
cases specific to one particular use. It is worth noting that
price-stable tokens are only one subset of possiblities within the
smartcoin framework.
