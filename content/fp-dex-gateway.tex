\label{sec:gateway}

The roles that traditional exchanges perform today encompass:

\begin{enumerate}
 \item \label{it:g1} Receiving crypto-currency and issuing IOUs.
 \item \label{it:g2} Receiving fiat and issuing IOUs.
 \item \label{it:g3} Redeeming IOUs.
 \item \label{it:g4} Processing an order book.
\end{enumerate}

Each of these stages requires a high degree of trust and direct counterparty
risk, because they involve an IOU from the exchange. To get the best
liquidity and lowest spreads requires a large and active order book, and this
means that most people gravitate toward a few core exchanges, leaving everyone
exposed to the same counterparty risk.

Moving money into or out of an exchange often incurs a significant time delay,
which means that active traders must keep their funds on the exchange. This
magnifies the amount of risk to users of the exchange. It also magnifies the
risk to all users in the crypto-currency ecosystem. Each large security breach
results in significant sell pressure, from both the thief looking to cash in
their loot, and from regular users hoping to sell before the thief.

\medskip

With the separation of powers, we only need gateways that perform
\cref{it:g1}, \cref{it:g2} and \cref{it:g3} while order book processing and
storage of account balances are managed by the BitShares protocol/network. An
entity issuing and redeeming IOUs for an other asset in BitShares is called a
\emph{gateway}. In contrast to central exchanges, the IOUs are send directly
to the wallet of the costumer directly and are his under full control (see
\cref{sec:uia:restrictions}).

Many gateways prefer the low-risk approach of one-for-one redemption and will
simply allow the GatewayUSD to float against BitUSD with a small but variable
spread in the market. Users then pay a small variable conversion cost as they
exit from BitUSD to fiat USD through GatewayUSD.

On the other hand, many users will want a direct conversion from BitUSD to fiat
USD. In this mode of operation, the gateway takes care of providing all of the
liquidity within a fixed percentage transaction fee. The gateways then
compete on offering the lowest possible spread.

Once this happens, BitUSD is effectively as good as USD with a small fixed
conversion fee. This fee will likely be no more than the withdraw and deposit
fees that current exchanges charge. At that point, BitShares will be a fully
operational exchange with many banking partners and no limits. At no point in
time will user deposits ever be subject to default or confiscation by an
exchange or gateway. A truly decentralized exchange will have been realized,
and the original vision of BitShares completed.
