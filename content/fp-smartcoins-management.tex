Every smartcoin is represented by a unique symbol on the blockchain. Together
with the parameters this symbol contains a management account (often
misleadingly referred to as \emph{issuer}). This management account has control
over the parameters and also distinguishes two types of smart coins, namely the
\emph{bitassets} from \emph{privatized bitassets}.

Regular BitAssets are smartcoins that are managed by the BitShares Committee
Account and thus, ultimately, by the BTS holders that vote the BitShares
Committee.  The Committee constitutes at least 11 members that jointly control
the committee-account. As a consequence of this being community driven, the
smartcoins are usually prefixed by \texttt{bit} in most user interfaces. It is
worth noting that bitUSD is such a community-driven smartcoin that uses the
symbol name \texttt{USD}, internally. Same holds true for bitCNY and others.

In contrast, privatized BitAssets are those smartcoins that are managed and
maintained by independent 3rd parties. For sake of neutrality, we do not go
into further detail about these in this paper.
