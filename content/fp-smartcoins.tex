\label{sec:mpa}
%%%%%%%%%%%%%%%%
% Smartcoins
%%%%%%%%%%%%%%%%
In today's world, crypto-currencies are unique because they are the only type
of digital currency that does not represent a corresponding counterparty
liability. Instead, they are \emph{fungible} and \emph{decentralized} tokens, whose
value is derived from the amount of practical utility (or potential future
utility) perceived by the network of users that support and trade in them.

Not surprisingly, most crypto-currencies suffer from high levels of price
volatility due to many complex factors, such as constantly shifting
public perception and highly speculative and unregulated markets.
Although professional traders tend to appreciate this volatility, so far
it has hindered the widespread adoption of cryptocurrency as a
\emph{practical payment solution}. A crypto-currency that has the
properties and advantages of Bitcoin, but is also capable of maintaining
price parity with a globally adopted currency (e.g. U.S. dollar), would
have incredible high utility for convenient and fast e-commerce.

The BitShares Blockchain offers a solution to this problem by introducing
\emph{smartcoins} - a framework for collateralized, counterparty risk-free loans
that enable token that highly correlate with the valuation of an
underlying asset, such as U.S. Dollar. Said framework is realized by a set of smart contracts
on the blockchain. 

Most prominent example of smartcoins on the BitShares Blockchain are 
\emph{bitUSD} and \emph{bitCNY}, which are smartcoins specific to one particular use
- namely to realize price-stable tokens. It is worth noting that that
price-stable tokens are merely one subset of possiblities within the
smartcoin framework. In literature, smartcoins are also often referred
to as \emph{bitassets} or \emph{market pegged asset}.