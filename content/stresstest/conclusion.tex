With our testing, we have seen that the code base is much more robust
than we had expected. The block creation and block validation is
sufficiently fast so that we can run a blockchain with synchronous block
confirmations times of \SI{1}{s} without validators dropping out too
much.

The Peer-2-Peer code seems to be working nicely but results show that it
could be one of the bottlenecks currently preventing us from going
further. The computational resources of our validators have more than
enough back-off for higher throughputs but the networking code was not
able to provide sufficient data (e.g. transactions) to raise it during
our stress-test.

We have further identified another bottleneck with respect to transaction
production. Signing transactions takes more resources and we will
prepare properly for our next stress-test.

To conclude, the current software stack still has a few limitations and
edges that need to be optimized in order to improve scalability further,
but the foundation has been designed in such a way that it allows for
even higher throughput. Our current limitations are purely in the
implementation and networking aspects than in the software and protocol
architecture.
