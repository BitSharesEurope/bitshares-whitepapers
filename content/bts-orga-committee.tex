Since Bitcoin struggled to reach a consensus about the size of their blocks,
the people in the cryptocurrency space realized that the governance of a DAC
should not be ignored. Hence, BitShares offers a tools to reach on-chain
consensus about business management decisions.

The BitShares blockchain has a set of parameters available that are subject of
shareholder approval. Shareholders can define their preferred set of parameters
and thereby create a so called \emph{committee member} or alternatively vote
for an existing committee member. The BitShares committee consists of $C$
\emph{active} committee members.

For each business parameter the protocol will calculate the difference between
up- and down-votes ($v_\text{pro}-v_\text{con}$) for each active committee
member and then take the median of the top $C$ active members:
\begin{algorithmic}
 \State // Derive active C committee members
 \For{ i : active committee members }
  \State member weight: $w[i] \gets v_\text{pro}-v_\text{con}$
 \EndFor
 \State $\text{members} \gets \Call{sort}{w}$
 \State $\text{active} \gets \text{members}[0 \to C]$

 \State // For each Parameter: derive median of active members
 \For{ parameter : parameters }
   \State $p \gets \Call{GetParameters}{\text{active}, \text{parameter}}$
   \State $x = \operatorname{sort}(p[i])$
   \State $\tilde p =\begin{cases}
                        x[\frac{C+1}{2}]                                               & C \text{ odd}\\
                        \frac {1}{2}\left(x[{\frac{C}{2}}] + x[\frac{C}{2} + 1]\right) & C \text{ even.}
           \end{cases}$
  \State $\text{parameter} \gets \tilde p$
 \EndFor
\end{algorithmic}
Since, $C$ is a parameter as any other, the shareholders decide for the size of
the committee.

The BitShares ecosystem has a set of parameters available that are subject of
shareholder approval. Initially, BitShares has the following blockchain
parameters:
%
\begin{description}[leftmargin=4em,style=nextline]
 \item[{fee structure}:        ] fess that have to be paid by customers for individual transactions
 \item[{block interval}:       ] i.e. block interval, max size of block/transaction
 \item[{expiration parameters}:] i.e. maximum expiration interval
 \item[{witness parameters}:   ] i.e. maximum amount of witnesses (block producers)
 \item[{committee parameters}: ] i.e. maximum amount of committee members
 \item[{witness pay}:          ] payment for each witnesses per signed block
 \item[{worker budget}:        ] available budget available for budget items (e.g. development)
\end{description}
Please note that the given set of parameters serves as an example and that the
network's parameters are subject to change over time.

\medskip

Additionally to defining the parameters any active witness can propose a
protocol or business upgrade (i.e. hard fork) which can be voted on (or
against) by shareholders. When the total votes for the hard fork are greater
than the median witness weight $w$ then the hard fork takes effect.
