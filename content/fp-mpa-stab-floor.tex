In order for BitUSD to be accepted as being equal to \$1.00 for the purposes of
setting prices and online shopping, it only needs to maintain a \emph{floor} of
\$1.00. If it can maintain a floor of \$1.00, then merchants can accept it,
know their margins are safe, and that they are \emph{not exposed to currency
risk}. In order to enable a guaranteed floor, all BitUSD can be \emph{force
liquidated} at a trustworthy price feed\footnote{Price feeds are published by
\emph{delegates} that have shareholder approval. See~\cref{sec:feeds}.}. If
this rule is present, then those who create the BitUSD must sell it at a price
that properly accounts for this risk of so called \emph{forced settlement}.
This means that at almost all times new BitUSD will only enter circulation when
there is a buyer willing to pay a premium for a guaranteed floor.

As we will see, since USD holders can initiate settlement, there is no need for
artificial forced covering every 30 days. This relieves shorts of risk, helps
increase short demand, and keeps the price of BitUSD near the floor.

Also note that for reasons of easier description in this paper, the terms
\emph{SmartCoins} and \emph{Market Pegged Assets} are now synonym to BitAsset
2.0.
