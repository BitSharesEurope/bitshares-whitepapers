After investigating the throughput of the blockchain, we also tested the
Peer-2-Peer network and capability to deal with different block confirmation
times. As a reminder, the technology tested here is capable of changing
different blockchain parameters on the fly. One of these parameters is the block
confirmation time which represents the time between the blocks in the
synchronous consensus mechanism delegated proof-of-stake (DPOS).

\begin{figure}[!htp]
 \centering
 \includegraphics[width=\linewidth]{figures/stress-test-block-time-full.png}
 \caption{Max ops/s during the stress-test}
 \label{fig:time-full}
\end{figure}

In \cref{fig:time-full}, we can see the actual block confirmation time as
tracked by the blockchain. These values have been obtained from each block
header which also contains the time stamp. Little variations are to be expected
due to slightly inaccurate synchronization of the local time, while large
deviations indicate one or multiple validators to have missed their blocks. The
graphs shown in the figure are averaged over \SI{100}{blocks} to reduce the
noise.

We can see that, during the stress-test, multiple different block confirmation
times were tested, starting with \SI{5}{s}, going down to \SI{2}{s}, then
up to \SI{10}{s} until we decided to also test \SI{1}{s} block confirmation
time. Unfortunately, we observed some witnesses not producing during large
parts of the stress-test which is why the averages are above the target rates
most of the time. For our next stress-test we plan to be more strict in
replacing witnesses when they miss block production blocks.

\begin{figure}[!htp]
 \centering
 \includegraphics[width=\linewidth]{figures/stress-test-block-time-1sec.png}
 \caption{Max ops/s during the stress-test}
 \label{fig:time-1se}
\end{figure}

In \cref{fig:time-1sec} highlights the period when the block confirmation time
was reduced to \SI{1}{s}. We can see that most of the blocks were produced in
time with only up to 3 validators not having produced in time (i.e. the black
peaks going up to at most \SI{4}{s}). This came as a big surprise to us as we
had expected a higher block-miss rate all together.
