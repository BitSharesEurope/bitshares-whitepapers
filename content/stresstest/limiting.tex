In this section we briefly discuss the limiting factors we have identified
prior to the stress-test. For the sake of proper description, we first need to
look into the distinction of \emph{blocks}, \emph{transactions}, and
\emph{operations}.

\paragraph{Transactions and Operations}
In contrast to other blockchains, Graphene-based blockchain \emph{explicitly}
allow to bundle multiple \emph{operations} into a single transaction. Similar
to Bitcoin allowing to bundle multiple outputs, but with the option to execute
different smart contracts offered by the blockchain sequentially. The easiest
application is similar to Bitcoin's multi-send feature where a single user
sends bitcoins to many addresses. In the case of Graphene/BitShares, the user
would also sign a single transaction with a single signature but the
transaction would carry many \texttt{transfer} operations. Additionally, a
user may put a combination of \texttt{trade}/\texttt{transfer}/\texttt{borrow}
operations (or any other) into a single transaction to ensure that they are
executed sequentially.

As can be seen, the main difference between bundling many transactions into a
block and bundling many operations into a transactions is that in the latter
case, the operations are guaranteed to be executed sequentially and that only a
single authorization (e.g.\ signature) needs to be validated and verified.

This distinction is important as our test results will clearly distinguish
between the throughput of \emph{operations} and the throughput of
\emph{transactions}.

\paragraph{Signature Verification}
Similar to other blockchain technologies (i.e. Bitcoin), a single block may
contain multiple transactions. Each transaction is signed by one or multiple
entities and thus requires the validators to perform elliptic curve signature
verification as well as public key recovery for authorization.

From a scalability point of view, the limiting factors are the time and
resources that are required to validate a signature for a given transaction and
the time and resources that are required to append a transactions to the
blockchain and execute its corresponding smart contract/operation.

Since the Graphene technology is built such that the signature validation can
be done independent of the blockchain's current database state, the signature
verification and key recovery can be trivially parallelized by means of a
cluster or the use of a graphics processing unit (GPU). However, as the
required software for this parallel verification was not available at the time
of the stress-test, we expect the throughput to be significantly higher once we
can be validated on a GPU.

For this reason, and because we want to see the limiting factors when applying
validated transactions to the blockchain, parts of our test scenario are
focused around producing transactions that carry hundreds of \emph{transfer}
operations with just a single signature to be verified.

\paragraph{Connectivity}
Obviously, a globally distributed network represents the worst case scenario
when it comes to latency between nodes limited mostly by the speed of light and
the size of the planet. A network that is hosted locally can result in better
overall performance but does not offer the same robustness and redundancy as a
global network.

\paragraph{Memory Requirements}
In previous tests, we have identified a significantly increasing need for
memory by the blockchain providing back-end software. This was mostly caused by
so called \emph{plugins} that take care of historic events, such as account
transactions history or the trading histories. Those plugins kept their data in
memory for performance reasons. Once we disabled these plugins, the validating
software only required a few hundred Megabytes of memory.

\paragraph{Transaction Production}
In order to stress-test a network, we need to produce actual stress which is not
easy to achieve at the scale that we needed for our tests. Furthermore, we
wanted to simulate a more or less realistic scenario in which multiple
independent parties distributed on the globe produce transactions, each
transaction trying to make it into the next block.

For this reasons, we have asked the BitSharesTalk community to participate in
the stress-testing not only by means of providng validation nodes across the
planet, but also to produce the required stress.

\paragraph{Peer-2-Peer Networking}
The original codebase of Graphene has a hard-coded limit in its Peer-2-Peer
networking code (not part of the blockchain consensus protocol), that prevented
us from reaching more than \SI{1000}{tps} due to restriction in the amount of
transactions being obtained through the Peer-2-Peer network. After increasing
this limit way beyond what we wanted to achieve in our stress-test, we allowed
the validating nodes to obtain up to thousands of transactions per second from the
Peer-2-Peer network.

However, the Peer-2-Peer networking code has been optimized for a block
interval of about \SI{10}{s} and thus leaves a lot of room for optimizations
when talking about high-throughput at low latencies.
