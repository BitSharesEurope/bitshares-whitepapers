We define an electronic coin as a chain of digital signatures. Each owner transfers the coin to the
next by digitally signing a hash of the previous transaction and the public key of the next owner
and adding these to the end of the coin. A payee can verify the signatures to verify the chain of
ownership.

The problem of course is the payee can't verify that one of the owners did not double-spend
the coin. A common solution is to introduce a trusted central authority, or mint, that checks every
transaction for double spending. After each transaction, the coin must be returned to the mint to
issue a new coin, and only coins issued directly from the mint are trusted not to be double-spent.
The problem with this solution is that the fate of the entire money system depends on the
company running the mint, with every transaction having to go through them, just like a bank.
We need a way for the payee to know that the previous owners did not sign any earlier
transactions. For our purposes, the earliest transaction is the one that counts, so we don't care
about later attempts to double-spend. The only way to confirm the absence of a transaction is to
be aware of all transactions. In the mint based model, the mint was aware of all transactions and
decided which arrived first. To accomplish this without a trusted party, transactions must be
publicly announced [1], and we need a system for participants to agree on a single history of the
order in which they were received. The payee needs proof that at the time of each transaction, the
majority of nodes agreed it was the first received.
