\newcommand*\justify{%
  \fontdimen2\font=0.4em% interword space
  \fontdimen3\font=0.2em% interword stretch
  \fontdimen4\font=0.1em% interword shrink
  \fontdimen7\font=0.1em% extra space
  \hyphenchar\font=`\-% allowing hyphenation
}

As in other cryptocurrencies, the public ledger of BitShares is built and
stored in a linked series of blocks, known as a blockchain.

The ledger provides a permanent record of transactions that have taken place,
and also establishes an order in which transactions have occurred. Hence, every
\emph{content} of the blockchain can be assigned an permanent and unique
identifier in form of a scalar number.

Every full node in the BitShares network stores a full copy of this blockchain
and can verify its validity and the evaluate new blocks.

Every block contains
\begin{itemize}
 \item a reference to the previous block,
 \item a timestamp,
 \item a hash of a secret,
 \item the secret of the previous hash,
 \item a set of transactions, and
 \item a signature by the block producing authority
\end{itemize}

As will be discussed in \cref{sec:consensus}, the consensus mechanism allows
for synchronous block production with constant block confirmation times, e.g.,
one block every \SI{5}{\second}.

Since the blocks mainly embrace costumer transactions but has to perform time
intensive tasks, or execute rare events from time to time, some actions such as
reenumeration of blockchain-based votes and rare events such as newly
registered block producers (witnesses) are carried out more rarely but still on
a frequent so called \emph{maintenance interval}.

The following parameters are associated with the blockchain operations and are
subject to shareholder consensus:
%
\texttt{\justify
 block-interval,
 maintenance-interval,
 maximum-transaction-size,
 maximum-block-size,
 maximum-witness-count,
 maximum-committee-count.
}
