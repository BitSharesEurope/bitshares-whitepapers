We see that the seed allocation (initial distribution) of BitShares, which took
place over a 1 year period, from November 2013 to November 2014, was achieved
by sharedropping 47\% to BitShares PTS and another 47\% to BitShares AGS. This
way, the full, fairness was defined by equal opportunity and in the case of BTS
we have distributed \emph{fairly} by CPU mining of PTS while, alternatively,
everyone had an additional equal opportunity by contribute to AGS.

Having attracted two different groups of investors with a mined crypto token
via PTS and a donation based book of donors via AGS, everyone had a chance to
participate and be rewarded with stake in the genesis block of BitShares 1.0.
This genesis block solely consisted of AGS and PTS holders on a 50\%/50\% ratio
such that the BTS tokens initially issued by this genesis can be considered
\emph{well distributed}.

The other 6\% are set aside to secure the future of BitShares and funds its
development and operational costs. In practice, they are put into the so called
\emph{reserves pool} that no one has control over except the BitShares
protocol. In contrast to many other crypto-currencies, every shareholder has a
say as to how these funds are spend (see \cref{sec:bts:revenue:expenses}).
