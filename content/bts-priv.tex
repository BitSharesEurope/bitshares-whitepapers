% This chapter is mostly a guess!

BitShares 2.0 makes use of confidential transactions (by means of blind
signatures~\cite{blindSigPaper}) in combination with stealth addresses to
protect costumers privacy while keeping scalability and performance.

In \cite{blindSigPaper}, a scheme was proposed that allows generating a blind
signature compatible with the existing Bitcoin protocol. There, the client
requests a set of parameters from a 3rd partiy (e.g. a \emph{signing server})
and synthesizes a public key to use in a transaction. To redeem the funds, the
client transforms the hash of the transaction (\emph{blinds}), sends to the 3rd
party to sign and then transforms the signature (\emph{unblinds}) to arrive at
a valid ECDSA signature. The signed transaction is published revealing the
synthetic public key and the unblinded signature. Hence, the 3rd party cannot
learn about its participation neither from the public key nor from the
signature.

The novelty of the scheme is that unlike the original Chaum blind signature
scheme, this approach does not allow anyone to prove that the signing party
signed a particular message, but instead provides a much stronger privacy: the
resulting signature, public key and the message are all completely unlinkable
to the signing party.

For BitShares, it allows to have \emph{someone} else sign something without
them being able to prove they were the one who signed it which is useful for
multi-signature/two-factor on confidential transactions.

In combination with stealth addresses, we can now derive a \emph{one-time
public key} that can be signed by \emph{someone} knowing that this
\emph{someone} cannot use any knowledge of the one-time public key to link back
to us.
