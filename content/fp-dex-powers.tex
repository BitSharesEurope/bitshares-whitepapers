There is no reason why the same entity needs to be responsible for
\emph{issuing IOUs} and for \emph{processing the order book}. It is only
because these two roles are combined that we have a tendency toward
centralization in the Bitcoin exchange space. Since, both sides of any order
book consist of amount/price pairs, we can easily store these information in a
public ledger, such as a blockchain. If we want to create a decentralized
exchange then the first step is to move the order book on to the blockchain so
that everyone can see (and audit) it. 

Since, the blockchain allows users to trade, for example, BitstampUSD against
BitfinexUSD, in order to easily move funds from one gateway to another, users
could even trade BitstampUSD against BitstampBTC or BitstampUSD vs BitfinexBTC.

In this model, traditional exchanges merely become \emph{gateways} that receive
fiat and issue GatewayFiat as an UIA on the blockchain. Later, they receive
GatewayFiat, execute a wire transfer and then burn the GatewayFiat. These
gateways can make their money entirely on transaction fees and from a
percentage of the trading fees similar to their current model.

Unfortunately, simply moving the order book to the blockchain is not enough,
because the market will naturally centralize around a few gateway IOUs and the
markets for them. BitstampUSD is not fungible with BitfinexUSD because they
have different trust profiles and regulatory considerations. Any of these IOUs
are subject to default just like the IOUs that currently exist on the
exchanges' internal databases. 

What we need to do is move the trust from individual issuers to the blockchain
itself. Hence, we have the bitUSD which is backed by collateral and is
independent of governments, and trades for \$1 independent of any gateway. It
is also universal as you don't need to register anywhere to use bitUSD (or any
other market pegged asset). This core features sets us apart from Ripple and
has as software protocol as counterparty that does provably not cheat on your
funds.
