Decentralization gives BitShares robustness against failure. When a centralized
exchange is compromised, millions of dollars and thousands of users are
impacted all at once. In a decentralized system, any attack or failure would
impacts only a single user and their funds. Users are in control of their own
security, which is generally preferable to trusting a centralized entity.

Furthermore, since KYC/AML verification can be outsourced via whitelists, a
gateway that holds customers funds for fiat backed IOU's would not necessarily
need direct access to the customers' identities. This was an issue with
Mt.Gox~\cite{mtgox}, where thousands of customers' identities were stolen.

Since there is a fixed cost associated with attempting to hack an exchange or
an individual user, the difference between a centralized exchange and the DEX
is the size of the reward. If someone places a multi-million dollar bounty on
attacking a specific exchange, then you can expect a lot more effort to be put
into compromising that exchange than would be put into attacking your
individual account. 

Furthermore, within any centralized company multiple people usually have access
to customer funds. Likewise, most centralized exchanges end up depending upon
multiple people who share the responsibility of guarding the secret key that
controls the funds. If any one of them is compromised, everyone's funds are put
at risk. Because of this, being individually responsible for maintaining your
own secrets is a much safer option.

Access to funds in BitShares can be further secured by means of corporate
accounts that implement threshold signatures~\cite{bts:general,ripple:multisig}
and validate only those transactions which signature weights (e.g. the CEO has
more say than a worker) surpass a pre-defined threshold.
