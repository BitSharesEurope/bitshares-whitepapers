\label{sec:dexfee}
% \newcommand*\justify{%
%   \fontdimen2\font=0.4em% interword space
%   \fontdimen3\font=0.2em% interword stretch
%   \fontdimen4\font=0.1em% interword shrink
%   \fontdimen7\font=0.1em% extra space
%   \hyphenchar\font=`\-% allowing hyphenation
% }

In the BitShares ecosystem every operation is assigned an \emph{individual}
fee. These fees are subject to change. However, they are defined solely by
shareholder approval, thus each and every shareholder of the BitShares core
asset (BTS) has a say as to what the fees should be. If shareholders can be
convinced to reduce a certain fee and consensus is reached, the fee will be
reduced automatically by the blockchain\footnote{Changes of blockchain
parameters are proposed by members of the committee. These members are voted by
shareholders and improve the flexibility and reaction rate.}.

% The following fees are associated with the DEX and financial instruments:
% %
% %\texttt{\justify
%  transfer,
%  limit-order-create,
%  limit-order-cancel,
%  call-order-update,
%  fill-order,
%  asset-create,
%  asset-update,
%  asset-update-bitasset,
%  asset-update-feed-producers,
%  asset-issue,
%  asset-reserve,
%  asset-fund-fee-pool,
%  asset-settle,
%  asset-global-settle,
%  asset-publish-feed.
% %}
% %
% Some more fees are available on the protocol level but are not a subject of
% this paper. Which fees currently apply can be extracted from the blockchain and
% will certainly be put on a distinct information page of most wallet software.
