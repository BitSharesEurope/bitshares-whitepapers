In this section we briefly discuss the limiting factors we have identified
prior to the stress-test. For the sake of proper description, we first need to
look into the distinction of \emph{blocks}, \emph{transactions}, and
\emph{operations}.

Similar to other blockchain technologies (i.e. Bitcoin), a single block may
contain multiple transactions. Each transaction is signed by one or multiple
entities and thus requires the validators to perform elliptic curve signature
verification as well as public key recovery for authorization.

In contrast to other blockchains, Graphene-based blockchain \emph{explicitly}
allow to bundle multiple \emph{operations} into a single transaction. Similar
to Bitcoin allowing to bundle multiple outputs, but with the options to execute
different smart contracts offered by the blockchain subsequently. The easiest
application is similar to Bitcoin's multi-send feature where a single user
sends bitcoins to many addresses. In the case of Graphene/BitShares, the user
would also sign a single transaction with a single signature but the
transaction would carry many \texttt{transfer} operations. Additionally, a
user may put a combination of \texttt{trade}/\texttt{transfer}/\texttt{borrow}
operations (or any other) into a single transaction to ensure that they are
executed subsequently.

As can be seen, the main difference between bundling many transactions into a
block and bundling many operations into a transactions is that in the latter
case, the operations are guaranteed to be executed subsequently and that only a
single authorization (e.g. signature) needs to be validated and verified.

This distinction is important as our test results will clearly distinguish
between the throughput of \emph{operations} and the throughput of
\emph{transactions}.

From a scalability point of view, the limiting factors are the time and
resources that are required to validate a signature for a given transaction and
the time and resources that are required to append a transactions to the
blockchain and execute its corresponding smart contract/operation.

Since the Graphene technology is built such that the signature validation can
be done independent of the blockchain's current database state, the signature
verification and key recovery can be trivially parallelized by means of a
cluster or the use of a graphics processing unit (GPU). However, as the
required software for this parallel verification was not available at the time
of the stress-test, we expect the throughput to be significantly higher once we
can validated on a GPU.

For this reason, and because we want to see the limiting factors when applying
validated transactions to the blockchain, parts of our test scenario are
focused around producing transactions that carry hundreds of \emph{transfer}
operations with just a single signature to be verified.
