Decentralization gives BitShares robustness against failure. When a
centralized exchange is compromised, millions of dollars and thousands of users
are impacted all at once. In a decentralized system, any attack or failure
impacts only a single user and their funds. Users are in control of their own
security, which can be much better than any centralized entity.

Furthermore, since KYC/AML verification can be outsourced via whitelists, a
gateway that holds costumers funds in terms of IOU backing fiat does not
necessarily have direct access to your identity as was the case with
Mt.Gox~\cite{mtgox} where thousands of identities have been stolen.

Since, there is a fixed cost associated with attempting to hack an exchange or
an individual user, the difference is the size of the reward. If you place a
multi-million dollar bounty on attacking a specific exchange, then you can
expect a lot more effort to be put into compromising that exchange than would
be put into attacking your individual account.

Within a given company, multiple people usually have access to customer funds.
%
%You may have heard the expression, "Three can keep a secret if two are dead".
%
Currently, all centralized exchanges end up depending upon multiple people who
share the responsibility of guarding the secret key that controls the funds.
If any one of them is compromised, everyone's funds are put at risk. Because of
this, being individually responsible for maintaining your own secrets is the
only safe option.

Accessing funds in BitShares can be even more secured by means of corporate
accounts that implement threshold signatures~\cite{bts:general,ripple:multisig}
and validate only those transactions which signature weights (e.g. the CEO has
more say than a worker) surpass a pre-defined threshold.
