%## How it Works

Recurring Payments are implemented as a set of withdrawal permissions. Each
account can grant any number of withdrawal permissions to other accounts. A
withdrawal permission includes following properties:
\begin{enumerate}
 \item Start Date
 \item End Date
 \item Withdrawal Limit per Period
 \item Period Length (i.e. 1 month)
\end{enumerate}

Any asset type can be used in the withdrawal limit.

After a user grants the withdrawal permissions, the authorized account is
allowed to make one transfer per period of an amount up to the limit. If
there is insufficient funds then the withdrawal will fail. Withdrawal
permissions are designed to be a convenience for merchants and users, as they
do not represent a commitment to pay.

It is up to each merchant to initiate each withdrawal. The BitShares platform
does not automatically authorize the transfer of funds unless sufficient
signing authority has been reached.

%## Daily Withdrawal Limits

For security purposes, many banks place daily withdrawal limits on user
accounts. In the event that an account is compromised, a thief is limited in
the amount of damage that they can do. Withdrawal permissions enable users
to protect their BitShares funds in the same manner. To do so, a user creates
two accounts: savings and checking.

The savings account has keys kept offline where they are unlikely to be
compromised. Before placing the keys in cold storage, the savings account
authorizes the checking account to make a daily withdrawal of up to \$1000, for
example.

The checking account can then pull money out of savings at up to this limit,
per day, and then use those funds as needed. This gives the user confidence
that their losses would be limited if their account is compromised.

%## Scheduled Payments

As stated above, the withdrawal permission system does not automatically make
payments. However, BitShares has another feature which enables scheduled
payments: proposed transactions. At any time, a user can propose a transaction
to execute at a specific date and time in the future. If the transaction has
sufficient authorization (i.e. is properly signed by authorities) at the
specified time, then it will automatically be executed.

A merchant can use this feature, combined with withdrawal permissions, to
implement automatic payments after a one-time setup fee. In practice, it may
be cheaper for merchants to maintain their own scheduler to automate billing,
since the blockchain charges a fee to propose a transaction separately from the
transaction's own fees.
