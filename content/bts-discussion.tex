In general, BitShares has similarities and differences to most known
cryptocurrencies. As many others, BitShares is based on a blockchain that
stores and propagates transactions, i.e. user operations. Since, with DPOS,
computational resources are used solely for the purpose of transaction
propagation and confirmation, rather than wasteful computational work, the
block production interval has been reduced to a few seconds. Eventually, this
improves the over-all profitability of the DAC.

Additionally, we make use of \emph{named} accounts that can be registered on
the blockchain. Users no longer need to send money to an alphanumeric string
that can be copied incorrectly. Rather, funds can be sent as easily as sending
an email, and in the same fashion. Name registration allows for the
identification of who transactions are are originating with with no need to
manually create a contact account for a given address. Transactions may contain
a memo field that allow users to describe the nature of the transaction or
broadcast secure messages about the price of the current transaction fee. Since
BitShares 2.0 implements \emph{confidential transaction}, there is no longer a
need for \emph{mixing} or \emph{master nodes}. Transactions can be more private
iin BitShares than in Bitcoin, for example, with no additional work needed from
the user.

BitShares is a 100\% proof-of-stake system. This means it is a lot more
efficient (cost per security) than proof-of-work and therefore does not have to
dilute stakeholders/coinholders (there is a ~10\% yearly dilution of
Bitcoin-holders as per 2015 with Bitcoin and lowering this dilution would mean
to lower the security). Hence, the cost of securing the BitShares network is
merely a fraction of all transaction fees accumulated by the network. 

The job of the block producers is simple: include as many valid transactions in
your given block as possible and sign a single block. These Block producers
compete for the most approval in order to be allowed to produce blocks.
Shareholder votes are proportionate to the relative number of shares they own.
The BitShares DAC is \emph{completely} shareholder run. Now people can be hired
by the blockchain. Where coins like Bitcoin dilute to pay for network security,
BitShares takes these fees and directs them towards continual improvement of
the network and community. This helps insure BitShares will stay competitive in
its feature set. More details about the consensus scheme of BitShares can be
found in a separated whitepaper~\cite{}.

Recalling the initial distribution of BTS, it seem convincing to assume that
most alternative distributions are way more unfair and some disproportionately
favor their respective core developers. Since BitShares is a self-funded DAC,
it can \emph{pay} for its future development autonomously by dilution, if
shareholders reach an on-blockchain consensus by approval voting.

