The Graphene Platform has been developed by Cryptonomex specifically for
BitShares. It has undergone many changes and has been enhanced to stay on top of
blockchain technology. One of its public forks, BitShares, has been publicly traded
for over 2 years and has yet to show its maximum potential with respect to throughput.
For this reasons, we have deployed a testnet, that uses the very same technology that BitShares is built
on, with the specific purpose of testing algorithms, implementations, and also scalability.

Here, the term \emph{scalability} refers to the amount of transactions that can
be applied to the blockchain at scale. Several factors need to be taken into
account to correctly interpret any results obtained through testing. Among
these are the specification of the deployed servers (CPU/RAM) that produce the
blocks (validators), the interconnectivity of the nodes in the Peer-2-Peer
network, the round-trip times and latency between nodes, as well as the
geographical distance between nodes.

The number of validators, furthermore, does not affect the throughput as long
as all active validators can keep up with the requirements of the network. All
active validators are treated equally and are given a slot to produce their
blocks in each round~\cite{bts:general}. Increasing the number of validators
(a.k.a. witnesses) comes with an increased robustness against server failures
yet also results in a longer duration to reach transaction
finality~\cite{bts:general} which describes the absolute irreversibility of
transactions after 2/3 of all validators have approved a given block and its
transactions.

The goal of the public stress-test performed on the graphene-based testnet was
to identify the limiting factors and bottlenecks at scale with multiple parties
involved. We furthermore wanted to demonstrate scalability of current state of
the art blockchain technologies.
