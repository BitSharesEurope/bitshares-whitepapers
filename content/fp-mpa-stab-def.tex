Before we discuss how BitShares achieves price \emph{stability} we first need to
define what properties make a currency \emph{stable}.

In the U.S., for instance, the Federal Reserve (FED) has a mandate of
\emph{stable prices} and it is almost universally accepted that this is a good
mandate. The same holds true for the Euro with its stability being
\emph{controlled} by the European Central Bank (ECB). Mostly every
country/nation or federation applies a similar concept.

However, it is also widely accepted among many crypto-currency fans that, in
the case of the US dollar, the FED has failed at their mandate because of
monetary inflation and increase in the money supply. Since monetary inflation
is a sustained increase in the general level of prices, it is equivalent to a
decline in purchasing power of money. Hence, a dollar buys less and less over
time. As a result, we see the dollar losing 99\% of its \emph{purchasing power}
since the FED was founded in 1913~(see \cref{fig:monetarybase}). 

\begin{figure}[!htp]
 \centering
 \includegraphics[width=\linewidth]{figures/monetary-base}
 \caption{St. Louis Adjusted Monetary Base~\cite{ambsl}}
 \label{fig:monetarybase}
\end{figure}

The goal of this paper is not to propose a replacement of central banks but to
clarify the terminology in particular with regards to ``stable''
cryptocurrencies. Unfortunately, some people in the cryptocurrency space are
attempting to provide an alternative currency that can achieve the same
mandate. However, since price stability at its heart is the same as \emph{price
fixing}, this is a well known economic fallacy that crypto-currencies should
avoid.

% The goal of the FED price stability mandate is to mask the systematic theft of
% all increases in the production efficiency of the economy. 

Imagine a central bank managed to keep prices stable through their monetary
policy with 0\% price inflation over 20 years. Now lets assume that during this
same 20 years the advances in robotics and automation resulted in a 3x increase
in efficiency and thus there are now 3x as much food, cars, phones, houses,
etc. For the sake of this example we will assume the population is the same and
everyone has the same amount of money in the bank. You would normally expect
that everything would be 1/3 the price and that everyone would be able to
afford 3x their prior life style. But because of the central bank's
intervention they have managed to also increase the money supply by 3x.
% and distribute it in secret. The end result is that some people get a 1000x
% increase in life style while everyone else stands still.

% FIXME: not very neutral!
% We can conclude from this example that the mandate for price stability is
% mostly a goal meant to mislead the general public and mask theft from the lower
% and middle classes on a massive scale even at 0\% price inflation. For this
% reason, we do not want to bring this same mandate to crypto currencies but
% instead aim to free us from monetary enslavement.

We notice that the goals should not be \emph{price stability} nor should we
target a \emph{stable value} or \emph{purchasing power} (at least not yet). 
%
What we want to achieve instead is 
\begin{itemize}
 \item a \emph{predictable} price with \emph{reduced volatility}
 \item a somewhat reliable ability to \emph{predict the future value} of a token, and
 \item a unit of account that doesn't have any meaningful capital gains or
       losses for tax purposes.
\end{itemize}

Hence, price ``stability'' really means price \emph{predictability} within some
tolerance level. In the case of the U.S. dollar, a willingness to accept a 5\%
loss (in purchasing power) per year via deflation, demonstrates that
predictability is more important than stability~\cite{bm:stable:impossible}.
