Having defined \emph{operations}, we can now put these into a \emph{list of
operations} and construct a \emph{transaction}.
%
% \begin{verbatim}
% {
%    "ref_block_num": ...,
%    "ref_block_prefix": ...,
%    "expiration": [...],
%    "extensions": [...],
%    "operations": [...],
%    "signatures": [...],
% }
% \end{verbatim}
%
In addition to its operations, a transaction also consists of 
\begin{inparaenum}[(a)]
 \item an expiration date,
 \item a reference block number,
 \item a reference block prefix,
 \item a set of extensions, and
 \item a set of signatures to authorize each operation.
\end{inparaenum}

Each node (including witnesses) verifies that all requires signatures to
perform the given operations are present and valid prior to propagating the
transactions to the rest of the network and hence to the witness node
constructing the next block. If the transaction is included into a block it is
considered \emph{finally valid} or \emph{executed}.
