The general concept of a smartcoins is to mimic the dynamics of
\emph{collateralized loans}.

These loan contracts are implemented as smart contracts and strictly follow
pre-defined and publicly known behavior. In the case of our collateralized
loans, the smart contract becomes the lender and requires a security from the
borrow in form of another digital token to protect against loan default.

Hence, anyone can borrow from the blockchain by providing sufficient
collateral. The smart contract autonomously and transparently ensure that each
loaned unit in existance is backed by more than \SI{100}{\percent} (mostly even
more than \SI{175}{\percent}) of its value by the collateral.

Given the nature of smart contracts on a public blockchain, all actions are
transparent and auditable. Acountability is baked into the protocol.

What makes smartcoins unique is that they are free from counterparty risk even
though they resemble a loan backed by collateral. This is achieved by allowing
the network itself (implemented as a software protocol) to be responsible for
securing the collateral and performing settlements and margin calls to ensure
sufficient backing. This will be described in greater detail below.

Let's take bitUSD as an example. The parameters (see below) of bitUSD require
a minimum collateral of \SI{175}{\percent} in BTS tokens at all times. This
means that, in order to borrow \SI{1}{bitUSD} from the smart contract, one
needs to provide BTS worth \SI{1.75}{USD} as collateral. At the time the bitUSD
is borrowed, the BTS used as collateral are locked away and only returned to
the user when the debt is cleared.
