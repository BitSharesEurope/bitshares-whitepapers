Additionally, the Graphene technology allows users to propose a transaction
which requires approval of multiple accounts in order to execute.
These transactions are only partially valid and do not \emph{execute} until
they are completely valid.

The user proposes a transaction, then signatory accounts add or remove their
approvals from this operation. When a sufficient number of approvals have been
granted, the operations in the proposal are used to create a virtual
transaction which is subsequently evaluated. Even if the transaction fails, the
proposal will be kept until the expiration time, at which point, if sufficient
approval is granted, the transaction will be evaluated a final time. This
allows transactions which will not execute successfully until a given time to
still be executed through the proposal mechanism. The first time the proposed
transaction succeeds, the proposal will be regarded as resolved, and all future
updates will be invalid.

The common use-case would be similar to so called \emph{multi-signature}
transactions which must be signed by two parties. Classical crypto currencies
had the issue that such \emph{proposed} transaction had to be communicated on
separated channels until all required signatures have been collected.
With BitShares, it is no possible to propose a transaction on the blockchain
and have the required signatures be added by the respective parties.

The proposal system in combination with corporate accounts allows for
arbitrarily complex or recursively nested authorities. If a recursive authority
(i.e. an authority which requires approval of \emph{nested} authorities on other
accounts) is required for a proposal, then a second proposal can be used to
grant the nested authority's approval. That is, a second proposal can be
created which, when sufficiently approved, adds the approval of a nested
authority to the first proposal. This multiple-proposal scheme can be used to
acquire approval for an arbitrarily deep authority tree.
