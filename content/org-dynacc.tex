% ## Background

The ability to require multiple digital signatures for sensitive operations on
the blockchain is integral to the security of the platform. While a single
secret key may be compromised, multiple keys distributed over multiple
locations add redundant protections, which result in a far more secure
experience.

Competing blockchain systems suffer from the following shortcomings:

\begin{itemize}
 \item The M-of-N model cannot sufficiently reflect the management hierarchies
       of many real-life organizations.
 \item Equal weighting of M keys is not sufficient to express asymmetric
       ownership over an account.
 \item Coordination and signing must be done completely out-of-band.
 \item Keys cannot be changed without coordinating with all other parties.
 \item Signatures cannot be retracted while waiting on other parties.
\end{itemize}

%## Use Case
Multi-signature technology is all about permission management, and permissions
should be defined in terms of people or organizations rather than keys.
Consider an example company that is run by 3 individuals: Alice, Bob, and
Carol. Alice and Bob each own 40\% of the company and Carol owns 20\%. This
company requires 2 of the 3 principles to approve all expenses. You could
define the company in terms of keys assigned to Alice, Bob, and Carol, but what
if Alice wants to protect her own identity with a multi-signature check? Alice
opts to use a service provider that performs 2-factor authentication on every
action Alice makes. This protects both Alice and the company and the company
does not need to change its permission structure to accommodate Alice's choice
of 2-factor authentication provider.

%## Solution

In BitShares, we introduce a new approach to permissions based upon accounts
which are assigned globally unique IDs.

Under this system, it is possible to define an account that has no keys itself,
but instead depends solely upon the approval of other accounts. Those other
accounts can, in turn, depend upon the approval of other accounts. This
process forms a \emph{hierarchy} of accounts that must grant permission. Each
account can change its own permissions independently of any accounts above it
in the hierarchy, which is what makes the permissions \emph{dynamic}.

Each account defines its permissions as a set of keys and/or other account IDs
that are each assigned weights by the account holder. If the combined weight of
keys and/or accounts exceeds a threshold defined by the account, then
permission is granted.

The second solution is to include the partially signed transaction in the
consensus state and allow accounts to publish transactions that add or remove
their approval of the transaction. This simplifies the signing coordination
problem, enables people to change their mind before the threshold is reached,
and applies the transaction immediately upon receipt of the final approval.

The process for executing a transaction that requires multi-signature authority
is as follows:
\begin{enumerate}
 \item Someone proposes a transaction and approves it with their account.
 \item Other account holders broadcast transactions, adding their "Yes" or "No"
       to the proposal.
 \item When the proposed transaction has the approval of all accounts, it is
       confirmed.
\end{enumerate}

%## Owner and Active Keys
Every account is assigned \emph{two} authorities: \emph{owner} and \emph{active}.

\begin{itemize}
\item An authority is a set of keys and/or accounts, each of which is
      assigned a weight.
\item Each authority has a weight threshold that must be crossed before an
      action requiring that authority may be performed.
\item The owner authority is designed for cold-storage, and its primary role
      is to update the active authority or to change the owner authority.
\item The active authority is meant to be a hot key and can perform any
      action except changing the owner authority.
\item The motivating use case is a 2-factor authentication provider as a
      co-signer on the active authority, but not on the owner authority.
\end{itemize}

With this approach, a user can remain confident that their account will always
be in their control, and yet that control can be kept in cold storage where no
one can hack it. This means that a company account can require the approval of
its board of directors and each board member may in turn require 2 factor
authentication.

Anyone can rotate keys frequently without having to disturb the permissions on
the accounts of its users.

%## Gathering Signatures

One of the challenges that has made multi-signature approaches difficult to use
in the past is that the act of gathering the required signatures was entirely
manual, or required specialized infrastructure. Once a transaction is signed,
there is no ability to retract your signature, so the last party to sign gains
a slight advantage over the other parties. With deeper hierarchies, gathering
signatures becomes even more complex.

To simplify this process, a blockchain should manage the signature gathering
process by tracking the state of partially approved proposed transactions.
Under this process, each account can add (or remove) their permission to a
transaction atomically, without having to rely upon an outside system to
circulate the transaction. This becomes especially critical for hierarchies
that are arbitrarily deep.

In order to keep things computationally bounded, an individual transaction will
only traverse down two layers in a hierarchy. If more than two layers of
hierarchy are present, then an account will have to propose (create one
transaction) to approve a proposal (the other transaction). When the first
proposal (transaction) is approved, permission is then added to the second
proposal (transaction).

Under this approach, each individual pays a single transaction fee each time
they approve an action, and every action involves at most 1 signature
verification by the network. This process allows arbitrarily deep hierarchies
to be formed without exposing the permission system to vulnerability of
unbounded computation.

%## Scalability

In theory, accounts can form a hierarchy that is arbitrarily deep, and
evaluating that hierarchy can take an arbitrary amount of time. In practice, it
is unlikely that a single transaction will have signatures more than 2 levels
deep, which keeps them computation bounded. Anything that requires more than 2
levels is likely to involve many people, and would not be signed all at once.
Instead, it would use the built-in proposed transaction infrastructure, which
tracks partially approved transactions.

\begin{itemize}
\item With this approach, a board member can propose that his company approve a
      transaction.
\item This can be extended logically to propose, and account propose, to
      approve a transaction.
\item This process would collect transaction fees as all of the layers in the
      hierarchy gradually add their permissions, and
      at no time requires an unbounded calculation.
\end{itemize}

%## Cycles

It is possible for two accounts to require each other to approve a transaction.

Imagine account $X$ is created that requires $A$ and $Y$ to approve.
Imagine account $Y$ is created that requires $B$ and $X$ to approve.
The graph looks like this:
\begin{align}
 A &\rightarrow X \leftrightarrow Y\\
 B &\rightarrow Y \leftrightarrow X
\end{align}

$A$ proposes that $X$ spend \SI{1}{BTS} and waits for approval from $Y$.
$B$ proposes that $Y$ approve the proposal from $A$ and waits for approval from $X$.

There is no way to resolve this problem with a single approval from any party
due to the following reasons:

\begin{enumerate}
 \item Neither account can act without the other and thus nothing can be
       accomplished.
 \item Cycles don't have to be direct as in this case, they can involve
       arbitrarily long sequences and thus be non-obvious.
 \item If users create an approval cycle in the active authority then the
       owner authority can be used to break the cycle; however, if they construct a
       cycle in the owner authority and the active authority then the accounts
       involved in the cycle would be locked out.
 \item In practice client software can detect cycles and prevent them from
       being formed.
\end{enumerate}

%## Conclusion

Dynamic hierarchical threshold multi-signature permissions provides people and
organizations with a more natural way to express ownership and control
policies. This approach makes the system easier to use, and ultimately more
secure, than existing solutions.

%## Credits

The Ripple wiki has a documented, but unimplemented, proposal for a similar
Multisign feature~\cite{ripple:multisig} that was discovered independently.
