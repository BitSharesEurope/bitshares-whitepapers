Similar to most cryptocurrencies, there is a set of predefined operations that
can be performed on the blockchain. In contrast to Bitcoin, which uses a
technique called \emph{script} to describe operations that shall be performed
in a programmatic way using \emph{OP} codes, the BitShares network has a
predefined (but extensible) set of operations that a user may perform.

All operations end up on the blockchain eventually. Once they are validated and
confirmed by a witness by being included into a block, they are \emph{executed}
and update the state of the blockchain accordingly.

The release version of BitShares 2.0 comes with 
\begin{inparaenum}[(a)]
 \item transfer ops,
 \item limit-order ops,
 \item call-order ops,
 \item fill-order ops,
 \item account ops,
 \item asset ops,
 \item witness ops,
 \item withdraw-permission ops,
 \item committee-member ops,
 \item vesting-balance ops,
 \item worker ops,
 \item custom ops,
 \item assert ops,
 \item balance ops, and
 \item override ops.
\end{inparaenum}
However, since BitShares allows for shareholder approved, live protocol
upgrades, the set of operations can be extended and modified. 
% FIXME do we allows for modifications of existing ops?

On the blockchain level, each operation is assigned an individual \emph{id}
with a custom set of parameters for performance reasons.

% FIXME example
