In this section, we briefly describe which parameters are available to tune
custom smartcoins and their behavior in the market.

\begin{description}
 \item[short backing asset]: This defines the asset that has to be used for
       securing the loan. In other words, this represents the collateral asset.
       Once the smartcoin has been registered, this parameter may only be
       changed if the supply is zero.
 \item[feed lifetime]: The life time of a price feed as provided by oracles.
       Once a feed is older than this, it will be ignored. This ensure that
       feed producers need to publish feeds frequently.
 \item[minimum feeds]: The required minimum number of unique feed producers that
       published a valid price feeds to enable the smartcoin. If less than
       these are available, borrowing and margin calls (see below) are
       disabled.
 \item[force settlement delay sec]: Those that hold a smartcoin (e.g. bitUSD)
       can settle their smartcoin (see below) for their pro-rata share in the
       collateral.
       % This behaves similar to a debt equity swap if BTS = equity :D
 \item[force settlement offset percent]: This offset allows to reduce the
       percentage of collateral that can be obtained through settlement.
       If this is set to \SI{0}{\percent}, then a settlement of \SI{1}{bitUSD}
       will result in BTS worth \SI{1}{USD} (according to price feed) after the
       force settlement delay.
 \item[maximum force settlement volume]: This parameter limits the maximum
       amount of smartcoins that can be settled per maintenance interval.
\end{description}

These parameters are tied to the smartcoin and can be modified by its manager.
The created smartcoin is also called the \textbf{long position}.

\paragraph{Example}
Throughout this paper the example of the smartcoin bitUSD is used, which is one of
the prominent price-stable smartcoins. Its parameters require
a minimum collateral of \SI{175}{\percent} in BTS tokens at all times. This
means that, in order to borrow \SI{1}{bitUSD} from the smart contract, one
needs to provide BTS worth \SI{1.75}{USD} as collateral. At the time the bitUSD
is borrowed, the BTS used as collateral are locked away and only returned to
the user when the debt is cleared.
The owner of the smartcoin is the committee, a group of people elected by the BTS holders. Here, the
group of feed producers coincides with the group of block producers (i.e. \emph{witnesses}).
These $N$ trusted witnesses can be constantly reviewed as all feeds are put on
the blockchain. Hence, misbehaving witnesses can be identified, fired from
their revenue generating witness position and lose their reputation.

Another example of a widely adopted price-stable smartcoin is bitCNY, which could be equivalently used for 
the examples.