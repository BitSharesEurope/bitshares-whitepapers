As a platform, several financial instruments are already implemented or close
to completion at the time of writing of this paper. Namely, in this paper, we
are investigating market pegged assets or ``smart coins'' (a digital crypto
token with a ``stable price''), user-issued assets (as a general purpose crypto
token), and the decentralized exchange (that allows trading between any two assets
on the blockchain).
%, and the collateralized bond market (enabling peer-to-peer (P2P) lending).

%%%%%%%%%%%%%%%%
% Smartcoins
%%%%%%%%%%%%%%%%
In todays world, cryptocurrencies are unique in that they are the only digital
currency that is not someone else's liability. They are fungible,
decentralized, and as valuable as the network of users that support them.
Unfortunately, they suffer from very high volatility, because their perception
of value constantly changes as users enter and leave the ecosystem. Although
many professional traders appreciate this volatility, it prevents its adoption
as a payment solution.

The traditional approach to creating a stable asset is to accept deposits and
issue a digital token as a claim receipt (i.e. ``I owe you''/IOU). Under this
approach, the token is valued by the market as a dollar, discounted by any
credit risk associated with the issuer. This can work well for transactions,
but less well as a form of savings. History has repeatedly proven that issuers
eventually go bankrupt due to fraud, incompetence, or government intervention.

More recent approaches have used a cryptocurrency as collateral in a contract
for difference (CFD)~\cite{def:cfd}. Under this approach, two parties take
opposite sides of a trade, where one party is guaranteed price stability, and
the other party is granted leverage. This approach works as long as sufficient
collateral exists, and the contract can be settled by an honest 3rd party with
a price feed.

%%%%%%%%%%%%%%%%
% UIA
%%%%%%%%%%%%%%%%
Additionally, the BitShares platform provides a feature known as "user-issued
assets" (UIA) to help facilitate profitable business models for certain types
of services.  The term refers to a type of custom token registered on the
platform, which users can hold and trade within certain restrictions. The
creator of such an asset publicly names, describes, and distributes its
tokens, and can specify customized requirements, such as an approved whitelist
of accounts permitted to hold the tokens, or the associated trading and
transfer fees. These tokens allow for diverse use-cases such as, for instance,
ownership tracking, crowd fund raising, IOUs, coupons, and many more.

%%%%%%%%%%%%%%%%
% DEX
%%%%%%%%%%%%%%%%
BitShares provides a high-performance decentral exchange DEX, with all the
features you would expect in a trading platform. Any two assets that are
registered on the blockchain (MPA or UIA) may be traded against each other at
any time. The DEX can handle the trading volume of the NASDAQ, while settling
orders the second you submit them. With this kind of performance on a
decentralized exchange, there is no more need to risk funds in centralized
exchanges.

% %%%%%%%%%%%%%%%%
% % Bonds
% %%%%%%%%%%%%%%%%
% Last but not least, the BitShares bond market (currently not implemented fully)
% is an investment marketplace accessible to anyone that lets you earn interest
% with any of your asset, or take a short position using any other asset as
% collateral.

%%%%%%%%%%%%%%%
In the following we will discuss existing and upcoming financial instruments
implemented in the BitShares network. It is recommended to previously read
through the basic technological components of BitShares in the other white
papers \cite{}.
