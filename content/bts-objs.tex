Speaking abstractly, the BitShares blockchain does not distinguish between
assets, account, or operations. These terms do only exist outside the
blockchain. Instead, the blockchain deals with \emph{contextual objects} that
are associated with a given set of features, permissions, etc.

Hence, on the BitShares blockchains there are no addresses similar to Bitcoin,
but objects identified by a unique \emph{id}, a \emph{type} and a \emph{space}
in the form:

\begin{verbatim}
   space.type.id
\end{verbatim}

The reserved \emph{space} are
\begin{verbatim}
   enum reserved_spaces {
      relative_protocol_ids = 0,
      protocol_ids          = 1,
      implementation_ids    = 2
   };
\end{verbatim}

As an example, we have the following objects:
\begin{description}[leftmargin=4em,style=nextline]
 \item[1.2.15]   15th blockchain account
 \item[1.6.105]  105th blockchain witness
 \item[1.14.7]   7th blockchain worker proposal
 \item[2.1.0]    wallet dynamic global properties
 \item[2.3.8]    8th asset
\end{description}

A programmatic description of all fields can be found in the sources.
