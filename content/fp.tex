%%%%%%%%%%%%%%%%
% Smartcoins
%%%%%%%%%%%%%%%%
In today's world, crypto-currencies are unique because they are the only type
of digital currency that does not represent a corresponding counterparty
liability. Instead, they are \emph{fungible} \emph{decentralized} tokens, whose
value is derived from the amount of practical utility (or potential future
utility) perceived by the network of users that support and trade in them. Not
surprisingly, most cryptocurrencies suffer from high levels of price volatility
due to many complex factors, such as constantly shifting public perception and
highly speculative and unregulated markets. Although professional traders tend
to appreciate this volatility, so far it has hindered the widespread adoption
of cryptocurrency as a \emph{practical payment solution}.

One approach to creating a \emph{price-stable} asset would be for an issuer to
accept deposits in return for a digital token as a \emph{claim receipt} (an "I
Owe You"). With this approach, the token would trade in the market as having
the same value as the underlying asset, minus any perceived credit risk
associated with the issuer. While this approach may work well for settlements,
it is far less secure as an instrument for long term savings. History has
repeatedly proven that many issuers will eventually go bankrupt due to
incompetence, government intervention or outright fraud.

%%%%%%%%%%%%%%%%
% MPA
%%%%%%%%%%%%%%%%
BitShares has developed an alternative approach to creating price stable
digital assets by using a cryptocurrency as \emph{collateral} in a
\emph{contract for difference} (CFD)~\cite{def:cfd}. With this approach, two
parties take opposite sides of a trade, where one party is guaranteed price
stability, and the other party is granted leverage. This works as long as
sufficient collateral exists, and the contract can be settled by an honest 3rd
party with a price feed

BitShares is a counterparty-trust free platform for financial smart contracts
which operates over the internet, and offers a set of \emph{financial
instruments} that includes CFDs. These contracts are \emph{derivative
instruments}, and as such they fall under the wider definition of financial
instruments. Financial instruments can be defined as \emph{tradable} assets of
\emph{any} kind, including cash, proof of ownership receipts, or a contractual
right to receive or deliver an underlying instrument, commodity, option, etc.
Additionally, several other digital financial instrument tools are currently
available on BitShares, such as Market Pegged Assets (MPA) or ``SmartCoins''
which represent a \emph{derivative} with fiat currency, gold, or even other
cryptocurrencies as the underlying asset. These SmartCoins derive their value
from contracts based on the performance of the BitShares base token (BTS).
Smart coins will be presented in detail in \cref{sec:mpa}. 

%%%%%%%%%%%%%%%%
% UIA
%%%%%%%%%%%%%%%%
The BitShares platform also contains an flexible feature called ``user-issued
assets'' (UIA) which will help facilitate a wide range of profitable business
models based around certain types of services. A UIA is a type of custom token
registered on the platform, which users can hold and trade within certain
restrictions. The creator of such an asset can publicly name, describe, and
distribute its tokens, and can specify custom requirements such as an approved
\emph{whitelist} of accounts permitted to hold the tokens, or the associated
trading and transfer fees. These tokens allow for diverse use cases such as
ownership tracking, crowd fundraising, IOUs, coupons, and many more that will
be discussed in \cref{sec:uia}.

%%%%%%%%%%%%%%%%
% DEX
%%%%%%%%%%%%%%%%
In order to \emph{trade} financial instruments, BitShares provides a
high-performance \emph{decentralized exchange} (DEX), with all the features
expected of a professional trading platform (see \cref{sec:dex}). Any two
assets that are registered on the blockchain (MPA or UIA) may be traded against
each other at any time. Orders can be settled almost instantly at speeds of up
to 100,000 transactions per second. With this kind of performance on a
decentralized exchange, there is no longer a need for traders to expose their
funds to the risks of centralized exchanges.

In this paper we will discuss the financial instruments available in the
BitShares network as well as the DEX. Before continuing, we recommend that you
read through the basic technological components of BitShares in the other white
papers~\cite{bts:general} (more papers to be published shortly).
