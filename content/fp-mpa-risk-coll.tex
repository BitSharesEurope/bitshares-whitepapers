Market pegged assets maintain their price parity due to being backed by
collateral that has an established real world value. When the value of the
collateral falls, the system is designed to react by driving the internal asset
exchange to match the new real world exchange rate and trigger margin calls as
necessary. However, there exists a possibility that the underlying collateral
(BTS) drops in value so quickly the market pegged assets become
under-collateralized. Often termed a ``black swan event'', a sudden crash of BTS
value could prevent the system from adjusting in time. In this event, the full
amount of collateral is no longer sufficient to purchase the market pegged
asset back at the new real exchange rate. In such an event, assets may trade
below their face value. It is possible the market could recover if BTS regained
value. It is also possible the market would need to be ``reset'' and asset
holders forced to settle for BTS collateral worth less than the intended face
value of their assets. Under normal conditions, short term market movements,
spreads, and fees charged by exchanges may also affect the potential cost of
conversion into and out of market pegged assets.










%% BTS2 - lessons learned
Unnecessary Collateral Restrictions

All collateral above the maintenance collateral limit is effectively
meaningless when it comes to enforcing the peg. A black swan event occurs
whenever the least collateralized position is unable to buy enough BitUSD to
cover. At this point, all positions are force settled and any additional
collateral maintained by the shorts is returned to them. The only reason for a
short to provide additional collateral beyond the maintenance level is to avoid
being forced to cover at a loss during a short squeeze or to avoid being the
first to be force settled by a BitUSD holder.

Either the maintenance collateral level is sufficient or the system is
fundamentally unsound. Maintenance collateral only needs to be high enough to
cover any slippage as a result of a short squeeze. Rules in BitShares that only
allowed users to increase their collateral put extra risk on shorts and did
nothing to protect against a black swan. By relaxing this restriction, shorts
face less risk and can gain higher leverage which will enable them to sell
closer to the floor.
