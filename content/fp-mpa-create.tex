\label{sec:mpa}

A BitShares MPA can be viewed as a contract between an asset buyer seeking
price stability and a \emph{short seller} seeking greater exposure to BTS price
movement. The open source BitShares software program implements a decentralized
marketplace for MPA where all transactions are recorded on the shared block
chain ledger and the software enforces the market rules. This block chain based
marketplace is referred to as the \emph{decentral exchange} or \emph{internal
market} (c.f., \cref{sec:dex}) to distinguish from \emph{external markets} such
as websites that facilitate the exchange of government issued currencies with
crypto-currency. 

SmartCoins are tokens of a particular MPA (e.g. bitUSD), take the concept of a
contract for difference, and make the long side fungible. For the purpose of
this discussion, we will assume that the long side of the contract is BitUSD
and that the backing \emph{collateral} is BTS (the BitShares core asset).

In practice, bitUSD are created on the BitShares blockchain when a BTS holder
asks the network for them by handing over \emph{collateral} to the network
essentially locking them in a contract for difference.

The collateral is only returned to the short seller when assets are purchased
back from the market and effectively destroyed to fulfill the contract. This is
referred to as \emph{covering a short}. If the value of the collateral relative
to the current price of the market pegged asset falls below a certain margin of
safety the assets can be automatically repurchased from the market before
collateral becomes insufficient. These rules create systemic demand for market
pegged assets while allowing them to remain fungible. To protect your contract
against \emph{margin calls} (automated, network initiated force settlement of
your contract at the price feed), you should at least maintain the so called
\emph{maintenance collateral level} at all times. Hence, the collateral only
needs to be high enough to cover any slippage as a result of a short squeeze.

Hence, at the moment of creation, the position of the \emph{shorter} has not
changed at all because he can directly cover the short position using the
bitUSD to gaining back his BTS used as collateral.

In summary, the following set of market rules apply to all market pegged assets
(for the sake of simplicity, we here focus on the MPA bitUSD):
\begin{itemize}
 \item Anyone with BitUSD can settle their position within an hour at the feed
       price.
 \item The least collateralized short positions are used to settle the
       position.
 \item The price feed is the median of many sources that are updated at least
       once per hour.
 \item Short positions never expire, except by hitting the maintenance
       collateral limit, or being force-settled as the least collateralized at the
       time of forced settlement (see point 2).
 \item In the event that the least-collateralized short position lacks enough
       collateral to cover at the price feed, then all BitUSD positions are
       automatically force settled at the price of the least collateralized short.
\end{itemize}

A simple metric for testing the validity of our claim that 1.00 BitUSD is
always worth at least \$1.00 is to demonstrate that, if you can find someone
willing to sell 1.00 BitUSD for \$1.00, that it would be the cheapest option for
buying BTS. This means that 100\% of the buying demand for BTS would be
available to give liquidity to BitUSD holders as a priority over BTS holders.

While the rules are simple, the consequences are less obvious. Let's analyze
this from the perspective of the various players and attempt to prove that this
condition is met.
