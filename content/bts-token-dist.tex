BitShares has set an example of a \emph{social agreement} by establishing its
own \emph{sharedropping} standards. The idea behind sharedropping is that any
future chain will always benefit by choosing to align itself with the ones who
worked hard at making the technology possible.

Hence, the seed allocation (initial distribution) of BitShares, which took
place over a 1 year period, from November 2013 to November 2014, was achieved
by sharedropping 47\% to BitShares PTS and another 47\% to BitShares AGS.
%
This way, the full, fairness was defined by equal opportunity and in the case
of BTS we have distributed \emph{fairly} by CPU mining of PTS while,
alternatively, everyone had an additional equal opportunity by contribute to
AGS~\cite{}.

The other 6\% are set aside to secure the future of BitShares and funds its
development. However, in contrast to many other cryptocurrencies, every
shareholder has a say as to who these funds are spend (see
\cref{sec:token:supply}).

The base tokens of BitShares \emph{2.0} will be distributed on a 1:1 basis
fully honoring the BTS tokens in the BitShares \emph{1.0} network.  For the
sake of completeness, the following paragraphs will describe the initial
distribution of BTS tokens in the aforementioned BitShares 1.0 network.

The following will discuss PTS and AGS in more detail.
