Historically we have stated that a blockchain becomes irreversible after one
round of block production with greater than 51\% participation. It turns out
that this metric is too fuzzy because of noise in how witnesses are ordered. In
an effort to provide stronger/absolute guarantees a new metric has been derived
that determines the exact point at which a particular block becomes
\emph{irreversible}. The algorithm to define the metric goes as follows:

Sort $N$ witnesses by the last block number they signed, then take the highest
block number that is lower than 66\% of all other witnesses. This will indicate
that said block has been confirmed by 66\% of all witnesses and is clearly
irreversible.

This particular metric is dynamic and can respond to changes in the order of
witnesses and is immune to situations where the network fragments into more
than two pieces. In the event of a major disruption users are guaranteed that
no block older than that number can ever be undone. 

If we had only 17 witnesses and 3 second block confirmation interval, then this
will take an average of 34 seconds. If we had 101 witnesses and 3 second blocks
then this will take an average of 3.3 minutes for block to be
\emph{irreversible},

Having this metric is important to give everyone in the network peace of mind
in the unlikely event that a software bug or network issue causes all witnesses
to fall out of sync and gives a clear measure of when they are considered back
in sync.

Anyone accepting transactions as final prior to the most recent irreversible
block is choosing to take some extra risk on their transaction.
