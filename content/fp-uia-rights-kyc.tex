To comply with Know Your Customer (KYC) laws the issuer must know every single
customer's real world identity. BitShares supports this by enabling both
\emph{whitelists} and \emph{blacklists} on the block chain. 

When an asset enables whitelists, no account may send, receive or trade that
asset without being on an authorized whitelist. Rather than requiring every
issuer to whitelist every customer separately, an issuer may specify a set of
identity verifiers that they trust to do this job. This allows issuers to
benefit from the network effect of validated users without having to do any
direct identity verification themselves.

With this feature, account funds can effectively be \emph{frozen} by removing
them from the whitelist. Of course this only affects those tokens of that
particular UIA. Additionally, the issuer may take back his assets from any
account, if required.

Note that these kind of administrative powers are available only for UIAs and
not for MPAs. Additionally an issuer may choose to indefinitely give up partial
of full control over each specific administrative power.
