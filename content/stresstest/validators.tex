The validators' job is to collect unconfirmed and pending transactions as
received through a Peer-2-Peer network, verify and validate them and publish
your approval over those transactions in form of a signed block that carries
those transactions.

Given that the Graphene testnet also uses Delegated Proof of Stake (DPOS) as
it's consensus mechanism, the validators (a.k.a. block producers) can be voted
in and out by means of the stake-based governance system~\cite{bts:general}.
This allows us to 
\begin{inparaenum}
 \item modify the number of validators
 \item replace failing validators by standby validators
 \item pick validators that are closer geographically (if location is known)
\end{inparaenum}
every maintenance interval. This interval is used to tally all votes made
since the previous maintenance block and is set to \SI{5}{min} on the
test-network.

At the beginning of the stress-test, we will have these validators active:

\begin{compactdesc}
 \item[blckchnd-x] Intel Xeon E3, 32GB RAM, bare bone, Germany
 \item[blckchnd-test] Intel Xeon E3, 32GB RAM, bare bone, Germany
 \item[jim.witness1] Intel Xeon® E5, 28-56GB RAM, Azure, South Korea
 \item[smailer-5]  Intel Xeon E3, 32GB RAM, Germany
 \item[init0] Intel Core i7, 32GB RAM, bare bone, Germany
 \item[init2] Intel Core i7, 32GB RAM, bare bone, Germany
 \item[lafona2] Intel Avoton, 16GB RAM, France
 \item[delegate.ihashfury] Intel Atom C2750, 32GB RAM, bare bone, France
 \item[f0x] Intel Xeon E5, 56GB RAM, Azure, USA
 \item[alpha-jpn] Intel Xeon E5, 56GB RAM, Azure, Japan
 \item[bravo-bra] Intel Xeon E5, 56GB RAM, Azure, Brasil
 \item[charlie-usa] Intel Xeon E5, 56GB RAM, Azure, USA
 \item[delta-gbr] Intel Xeon E5, 56GB RAM, Azure, UK
 \item[rngl4b] Intel Xeon E5, 32GB RAM, bare bone, Luxembourg
 \item[taconator-witness] Intel Xeon E5, 32GB RAM, Switzerland
 \item[arthur-devling] Intel Xeon E5, 39GB RAM, France
 \item[fr-blockpay] 
 \item[de-blockpay] 
\end{compactdesc}
