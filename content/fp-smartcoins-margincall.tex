\label{sec:margincall}

However, since markets do not have all their liquidity at the spot or feed
price the blockchain needs a back-off and start closing borrow positions
earlier to prevent a loan defaults. For smartcoins, the back-off factor is
called \emph{Maintenance Collateral Ratio (MCR)} and derives the
position-specific \emph{Margin Call Price} (MCP) by:
\begin{align}
 \MCP = \MCR \;\cdot\; \frac{\text{d}}{\text{c}}
\end{align}

The \MCR\ is a asset-specific parameter defined by the feed providers which can
change to adept to market conditions but is guaranteed to be above $1$. It is
derived from the median of the proposed \MCR s of the feed producers such that
only the majority of feed producers can manipulate this parameter. The \MCR\ implies 
a margin call price \MCP\, which is the price at which the positions gets liquidated automatically.

\paragraph{Example} A commonly agreed value for the \MCR\ is \num{1.75} or 175\%. With the numbers from the previous example
the the margin call price is derived as follows:
\begin{align}
 \MCP &= 1.75 \;\cdot\; \frac{\SI{100}{USD}}{\SI{2500}{BTS}} \\
      &= \SI{0.07}{USD \per BTS}
\end{align}
If the price feed goes below \SI{0.07}{USD \per BTS}, your position will be
automatically liquidated. A margin call is created by the smart contract, which means that the contract
itself will take the collateral and sell just enough of it to market
participants in the internal exchange to increase the \MCR\ back to its
required minimum~\cite{bsip31}.

However, it is worth noting that there is a protection of call positions
against short squeezes. A short squeeze implies that call positions are being
squeezed out of their positions by forcing them into margin and having the
contract sell the collateral to the market (usually at a loss for the owner of
the call position).

To protect shorts against short squeezes, margin calls only execute
\emph{up to} the short squeeze protection price (\SQP).

The \SQP\ is derived by
\begin{align}
 \SQP = p \;/\; \MSQR
\end{align}

with $p$ denoting the price in the form of \si{debt \per collateral}. 

\paragraph{Example}
With the numbers from previous example and a \MSQR\ of
\SI{110}{\percent}, margin calls would buy in the markets and pay up to
\begin{align}
 \SQP &= \SI{0.07}{USD \per BTS} \;/\; \SI{110}{\percent} \\
      &= \SI{0.077}{USD \per BTS} \\
      &\stackrel{\frown}{=} \SI{12.98}{BTS \per USD}\,.
\end{align}
During a margin call, the orders with the least collateral ratio will be called (liquidated)
first~\cite{bsip34}.
