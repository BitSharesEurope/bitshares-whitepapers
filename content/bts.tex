BitShares is a technology supported by next generation entrepreneurs,
investors, and developers with a common interest in finding free market
solutions by leveraging the power of globally decentralized consensus and
decision making. Consensus technology has the power to do for economics what
the internet did for information. It can harness the combined power of all
humanity to coordinate the discovery and aggregation of real-time knowledge,
previously unobtainable. This knowledge can be used to more effectively
coordinate the allocation of resources toward their most productive and
valuable use.

Bitcoin is the first fully autonomous system to utilize distributed consensus
technology to create a more efficient and reliable global payment network. The
core innovation of Bitcoin is the Blockchain, a cryptographically secured
public ledger of all accounts on the Bitcoin network that facilitates the
transfer of value from one individual directly to another. For the first time
in history, financial transactions over the internet no longer require a middle
man to act as a trustworthy, confidential fiduciary.

BitShares looks to extend the innovation of the blockchain to more industries
that rely upon the internet to provide their services. Whether its banking,
stock exchanges~\cite{btst:oldwp}, lotteries~\cite{play}, voting~\cite{fmv},
music~\cite{peertracks}, auctions or many
others, a digital public ledger allows for the creation of \emph{distributed
autonomous companies} (or DACs) that provide better quality services at a
fraction of the cost incurred by their more traditional, centralized
counterparts. The advent of DACs ushers in a new paradigm in organizational
structure in which companies can run without any human management and under the
control of an incorruptible set of business rules. These rules are encoded in
publicly auditable open source software distributed across the computers of the
companies' shareholders, who effortlessly secure the company from arbitrary
control.

BitShares does for business what bitcoin did for money by utilizing distributed
consensus technology to create companies that are inherently global,
transparent, trustworthy, efficient and most importantly profitable. Why and
how BitShares achieves a decentralized but profitable business is described in
more detail in a distinct paper~\cite{bts:structure}.

BitShares has went through many changes and has done its best to stay on top of
blockchain technology. Towards the end of 2014 some of the DACs were merged and
the X was dropped from "BitShares X" to become simply BitShares (BTS).

The next step in the evolution of BitShares was named \emph{Bitshares 2.0}, and
incorporates all of the feedback and lessons learned from the BitShares
stakeholders, partners, developers, marketers, and other community leaders
throughout a full year of research and development.

With the former BitShares 1.0, the core development team has closely controlled
the development and direction of BitShares. With BitShares reaching maturity at
version 2.0, the team is ready to remove the training wheels, and let the
direction of all future development be decided completely by stakeholder vote.

By utilizing a new worker voting system that will be included in BitShares 2.0,
the development will continue in whatever direction is approved by its
stakeholders. With this new structure, BitShares will be more robust, and
sustainable while being agile, flexible and adaptive to overcome unforeseen
hurdles of the future.

This paper is intended as an introduction to BitShares 2.0 and presents the
basic concepts of the peer-to-peer nature, the distributed public ledger in
form of a blockchain, and give a brief overview of the decentral consensus
mechanism applied to reach blockchain state consensus. We further discuss the
basic blockchain tokens (BTS), its distribution and usage in BitShares. We
also describe the wallet and operations with the network as well as outline the
functionalities of BitShares accounts.
