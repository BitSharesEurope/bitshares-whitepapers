\label{sec:consensus}

Consensus is the mechanism by which a subset of people decide upon unitary
rational action. The process of consensus decision-making allows for all
participants to consent upon a resolution of action even if not the favored
course of action for each individual participant. Bitcoin was the first system
to integrate a fully decentralized consensus method with the modern technology
of the internet and peer-to-peer networks in order to more efficiently
facilitate the transfer of value through electronic communication. The
proof-of-work structure that secures and maintains the Bitcoin network is one
manner of organizing individuals who do not necessarily trust one another to
act in the best interest of all participants of the network.

It is of importance to distinguish a democratic voting process in which every
citizen of a community has one and only one vote from a distributed consensus
mechanism in crypto-currencies hand over voting power either in relation to
hashing power (e.g. proof-of-work) or on a per stake basis (e.g.
proof-of-stake). In both cases, those that invest in the required
infrastructure to increase their voting percentage (i.e. by buying mining
hardware or stake) act as shareholder in a distributed community.

The BitShares community employs \emph{Delegated Proof-of-Stake} (DPOS) in order
to find efficient solutions to distributed consensus decision making.  DPOS
attempts to solve the problems of both Bitcoin's traditional proof-of-work
system, and the proof-of-stake system of Peercoin and NXT by implementing a
layer of technological democracy to offset the negative effects of
centralization. For historical reasons, the technology is still called
\emph{delegated} proof-of-stake even though what have been delegates in
BitShares 1.0 are now so called \emph{witnesses}.

In DPOS set of $N$ witnesses (formerly known as \emph{delegates}) sign the
blocks and are voted on by those using the network with every transaction that
gets made. By using a decentralized voting process, DPOS is by design more
democratic than comparable systems. Rather than eliminating the need for trust
all together, DPOS has safeguards in place the ensure that those trusted with
signing blocks on behalf of the network are doing so correctly and without
bias. A more detailed description about the distributed consensus mechanism as
well as a discussion how blockchain forking is prevented during attacks is
given in a separate paper~\cite{}. % FIXME link dpos paper

Additionally, each block signed must have a verification that the block
before it was signed by a trusted node. DPOS eliminates the need to wait until
a certain number of untrusted nodes have verified a transaction before it can
be confirmed.

This reduced need for confirmation produces an increase in speed of transaction
times. By intentionally placing trust with the most trustworthy of potential
block signers, as decided by the network, no artificial encumbrance need be
imposed to slow down the block signing process. DPOS allows for many more
transactions to be included in a block than either proof of work or proof of
stake systems.

In a delegated proof-of-stake system, centralization still occurs, but it is
controlled. Unlike other methods of securing crypto-currency networks, every
client in a DPOS system has the ability to decide who is trusted rather than
trust concentrating in the hands of those with the most resources. DPOS allows
the network to reap some of the major advantages of centralization, while still
maintaining some calculated measure of decentralization. Furthermore, once a
delegates has reached approval by shareholders, surpasses the threshold of the
most $N$ active witnesses, and, hence, is elected to actively participate in
the block production procedure, its power is \emph{equivalent} to all other
active witnesses. This system is enforced by a fair election process where
anyone could potentially become a delegated representative (witness) of the
majority of users.

Please note that DPOS has a recommended $1-2$ block confirmation versus
bitcoin's 6 block recommendation. DPOS is much more resistant against forks for
the following reasons:
\begin{itemize}
\item When a fork is produced it is
      very likely that all delegates have seen and processed your transaction and
      thus no alternative transactions can be broadcast and the next delegate is
      almost certain to include your transaction.  All delegates are much more
      trusted than miners.
\item The probability of a fork after a block has been produced is very low (<
      0.01\%) where as Bitcoin has 25 orphans in the last 22 days (about 1 per day in
      Dec 3,2014) which translates into 0.7\% of blocks are orphaned.
\item On normal operations, DPOS achieves a 100\% witness participation rate and when
      we are less than that it is more often because a delegate went offline and didn't
      produce a block than because they produced a fork. 
\item In BitShares 1.0 forks have almost always been resolved within 30 seconds. 
\end{itemize}

Assuming a 10 second block interval, Bitshares is mathematically over 70x less
likely to orphan after 1 block than Bitcoin after 1 block (10 minutes). After 3
blocks (30 seconds) any random orphan will have been resolved and the
probability of alternative chains is much lower than the 0.000001\% of Bitcoin.
By the time Bitcoin gets to .7\% orphan probability, BitShares has 60 blocks
which would have a probability of being orphaned of less than $10^{-120}$.

% FIXME: review these numbers
