Every smartcoin is represented by a unique symbol on the blockchain. Together
with the parameters this symbol contains a management account (often
misleadingly referred to as \emph{issuer}). This management account has control
over the parameters.

The BitShares Blockchain has one highlighted type of smartcoins, which are also
called \emph{BitAsset}. They constitute the following key aspects
\begin{itemize}
\item \textbf{Manged by the BitShares Committee}, which constitutes at least 11 
members that jointly control the committee-account, which are voted in by the BTS holders
\item \textbf{Price-fed by block producers}, which are also voted in by the BTS holders
\item \textbf{Backed by BTS} directly or backed by another smartcoin which is then directly backed by BTS
\end{itemize}
Such BitAssets are thus community driven and imply a certain level of trust. The
smartcoins are usually prefixed by \texttt{bit} in most user interfaces and also while addressing them,
but their actual on-chain symbol does not contain the \texttt{bit}, it is not even technically possible to create an asset that 
starts with \texttt{bit} (to prevent fraud). For example, the symbol of the price-stable BitAsset \texttt{bitUSD} is actually only \texttt{USD}.

This stands in contrast to privatized smartcoins that are managed and maintained by independent 3rd parties. For sake of neutrality, we do not go
into further detail about these in this paper.
