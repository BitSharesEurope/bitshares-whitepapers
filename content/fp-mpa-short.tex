When deciding a price at which to enter a short order, a trader must consider
the risk of being force settled. In this case, no trader will attempt to short
at or \emph{below} the price feed, because they could be forced to settle
\emph{at} the price feed. In fact, a smart trader would allow enough of a
spread to account for the risk of being forced to settle at a feed price that
was off by a small amount. Since feed errors are equally balanced between being
in the favor of the short and in the favor of the long (c.f. \cref{sec:feeds})
this leaves only the risk of being forced out of their position at an
inopportune time.

A short can minimize their exposure to the feed by providing enough collateral
to keep far above the least collateralized positions, and thus stay very
unlikely to be forced to settle at the feed or at an inopportune time.

\medskip
In practice, the only way new smartcoins enters circulation is if there is
someone willing to pay enough of a premium to convince a short to provide
guaranteed liquidity at the price feed on demand, while also covering the cost
of exchange rate risk. This premium will be higher for the backing
crypto-currency in a bear market, and will be lower in a bull market.

Someone who is short has only one way to exit their position: by buying
smartcoins off the market. This means that a short must also factor in the risk
that the premium may change. If a short position is entered in a bull market
with a 0.1\% premium, it may be forced to exit during a bear market with a 5\%
premium. In this event a short position is exposed to both exchange rate of the
dollar vs. BTS and the premium risk. On the other hand, a short entered during
a bear market with a 5\% premium may get to cover during a bull market with a
0.1\% premium.

For all intents and purposes, the premium is expected to move in the same
direction as the price, and thus speculators who only care about relative price
changes can ignore the premium.
