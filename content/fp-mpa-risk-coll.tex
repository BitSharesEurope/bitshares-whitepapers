Market pegged assets maintain their price parity due to being backed by
collateral that has an established real world value. When the value of the
collateral falls, the system is designed to react by driving the internal asset
exchange to match the new real world exchange rate and trigger force
settlements (i.e. margin calls) as necessary.

All collateral above the maintenance collateral limit is effectively
meaningless when it comes to enforcing the peg. Maintenance collateral only
needs to be high enough to cover any slippage as a result of a short squeeze.

However, there exists a possibility that the underlying collateral (BTS) drops
in value so quickly the market pegged assets become under-collateralized. Often
termed a \emph{black swan event} (c.f., \cref{sec:blackswan}), a sudden crash
of BTS value could prevent the system from adjusting in time. In this event,
the full amount of collateral is no longer sufficient to purchase the market
pegged asset back at the new real exchange rate. In such an event, assets may
settle at the price fees and are converted back into the underlying collateral
(BTS). This may expose costumers at the volatility risk of BTS. Under normal
conditions, short term market movements, spreads, and fees charged by exchanges
may also affect the potential cost of conversion into and out of market pegged
assets.
