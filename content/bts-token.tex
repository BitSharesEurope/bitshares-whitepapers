In the BitShares network the base token is called \emph{a BitShare} and carries
the abbreviation \texttt{BTS}. It is dividable into $10^5=\num{100000}$ sub-units.
% which are
% denoted as follows \note{proposal}:
% \begin{align*}
% \SI{1}{mises } &= \SI{0.00001}{BTS}\\
% \SI{1}{xennon} &= \SI{0.0001}{BTS}\\
% \SI{1}{oxyd  } &= \SI{0.001}{BTS}\\
% \SI{1}{graphn} &= \SI{0.01}{BTS}\\
% \SI{1}{epox  } &= \SI{0.1}{BTS}
% \end{align*}

In general, all properties of Bitcoin also apply to BTS, namely, they have
value, can be transfered on the blockchain and are secured by an Elliptic
Curve Digital Signature Algorithm (ECDSA) on the curve \texttt{secp256k1}.

In contrast to most crypto-currencies, BitShares does not claim to be a
currency but rather an \emph{equity} in a decentral autonomous company (DAC).
As a result, the market valuation of BitShares is free floating and may be as
volatile as any other equity (e.g. of traditional companies).

Nonetheless, BTS tokens can be used as \emph{collateral} in financial smart
contracts~\cite{bts:financial} such as market pegged assets and thus back every
existing smartcoin such as the bitUSD.

% FIXME what else?

The following subsection recapitulate the initial distribution and supply of
BTS.
